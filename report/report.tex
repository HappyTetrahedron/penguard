% This is based on "sig-alternate.tex" V1.9 April 2009
% This file should be compiled with V2.4 of "sig-alternate.cls" April 2009
%
\documentclass{report}

\usepackage[english]{babel}
\usepackage{graphicx}
\usepackage{tabularx}
\usepackage{subfigure}
\usepackage{enumitem}
\usepackage{url}


\usepackage{color}
\definecolor{orange}{rgb}{1,0.5,0}
\definecolor{lightgray}{rgb}{.9,.9,.9}
\definecolor{java_keyword}{rgb}{0.37, 0.08, 0.25}
\definecolor{java_string}{rgb}{0.06, 0.10, 0.98}
\definecolor{java_comment}{rgb}{0.12, 0.38, 0.18}
\definecolor{java_doc}{rgb}{0.25,0.35,0.75}

% code listings

\usepackage{listings}
\lstloadlanguages{Java}
\lstset{
	language=Java,
	basicstyle=\scriptsize\ttfamily,
	backgroundcolor=\color{lightgray},
	keywordstyle=\color{java_keyword}\bfseries,
	stringstyle=\color{java_string},
	commentstyle=\color{java_comment},
	morecomment=[s][\color{java_doc}]{/**}{*/},
	tabsize=2,
	showtabs=false,
	extendedchars=true,
	showstringspaces=false,
	showspaces=false,
	breaklines=true,
	numbers=left,
	numberstyle=\tiny,
	numbersep=6pt,
	xleftmargin=3pt,
	xrightmargin=3pt,
	framexleftmargin=3pt,
	framexrightmargin=3pt,
	captionpos=b
}

% Disable single lines at the start of a paragraph (Schusterjungen)

\clubpenalty = 10000

% Disable single lines at the end of a paragraph (Hurenkinder)

\widowpenalty = 10000
\displaywidowpenalty = 10000
 
% allows for colored, easy-to-find todos

\newcommand{\todo}[1]{\textsf{\textbf{\textcolor{orange}{[[#1]]}}}}

% consistent references: use these instead of \label and \ref

\newcommand{\lsec}[1]{\label{sec:#1}}
\newcommand{\lssec}[1]{\label{ssec:#1}}
\newcommand{\lfig}[1]{\label{fig:#1}}
\newcommand{\ltab}[1]{\label{tab:#1}}
\newcommand{\rsec}[1]{Section~\ref{sec:#1}}
\newcommand{\rssec}[1]{Section~\ref{ssec:#1}}
\newcommand{\rfig}[1]{Figure~\ref{fig:#1}}
\newcommand{\rtab}[1]{Table~\ref{tab:#1}}
\newcommand{\rlst}[1]{Listing~\ref{#1}}

% General information

\title{Penguard\\
\normalsize{Distributed Systems -- Project Proposal}}
\subtitle{Tracking objects to prevent them from being lost}

\numberofauthors{3}
\author{
\alignauthor {\normalsize Nils Leuzinger}\\
	\affaddr{{\normalsize 14-939-896}}\\
	\email{{\normalsize nilsl@student.ethz.ch}}
\alignauthor {\normalsize Nicole Thurnherr}\\
	\affaddr{{\normalsize 11-925-328}}\\
	\email{{\normalsize nicoleth@student.ethz.ch}}
\alignauthor {\normalsize Aline Abler}\\
	\affaddr{{\normalsize 14-920-979}}\\
	\email{{\normalsize ablera@student.ethz.ch}}
}

\begin{document}

\maketitle

\begin{abstract}
Don't you know this problem? You're just taking your penguin to party as usual, and all of a sudden it is stolen! 

We're tired of this. So now we're going to solve this problem once and for all. Penguard is an application that can prevent such things from happening by alarming us beforehand.
\end{abstract}

\section{Introduction}

Penguard is an application that allows users to monitor objects and informs them when these objects get too far away from them. This can be used as a theft alarm or as an aid to prevent forgetting objects.

While this principle is widely applicable, the idea stems from the need to guard a plush penguin mascot at an event and prevent it from being stolen. In the style of this particular use case, the objects to be monitored will henceforth be called ``penguins'', while the people monitoring them will be called ``guardians''.

Penguins can be guarded by several guardians at once, in which case it is sufficient for the penguin to be ``seen'' by one of them. Also, guardians can monitor several penguins at once. The guardians should know at all time which other guardians see which penguins.

We want to be able to use any kind of bluetooth device as a penguin. The only requirements should be that it supports bluetooth, and that it doesn't automatically turn off bluetooth after a while of inactivity, even when no device is paired to it.

Penguins should be able to detect when they are ``lost'' and act upon it. This obviously contradicts the goal to support any bluetooth device. We want to support both goals by adding an optional protocol through which penguins can communicate with the guardians. When the penguin supports the protocol, it can detect and act upon being lost, otherwise not.

The major challenges will be

\begin{itemize}
    \item Implementing a peer-to-peer protocol for the guardians
    \item Detecting bluetooth devices and ``tracking'' them without making more assumptions on them
    \item Implementing a protocol for the penguins to talk to the guardians
\end{itemize}

\section{System Overview}

A Penguard group consists of

\begin{itemize}
    \item One or more penguins being guarded
    \item One or more guardians guarding the penguins
\end{itemize}

Guardians can be part of at most one group at once. Penguins not supporting the Penguard Penguin protocol can be part of more than one group--this cannot be prevented. Penguins supporting the Penguard Penguin protocol can only be part of one group. It is theoretically possible for a second group to add it as a penguin not supporting the protocol, though.

The guardians communicate with each other via Internet using a peer-to-peer protocol (the Penguard Guardian protocol). They can discover each other by either using a Penguard Discovery server, or by broadcasting and receiving discovery packets over the local network. We want the Penguard Discovery server to be an optional component.

The guardians can detect whether a penguin is in range by using Bluetooth RSSI. That means guardians are not required to pair with penguins. 

Penguins can optionally implement the Penguard Penguin protocol, which is a Bluetooth Low Energy protocol. The guardians act as clients to the Penguins. The guardians will ping the Penguins regularly, such that the penguin can detect when it's lost using a timeout. The penguin should also advertize what kind of information it requires from the guardians, which the guardians then must send to it. This information can include phone numbers, email addresses or similar information and should allow for the penguin to contact the guardians when it is lost.

The guardians should be able to find out whether any given penguin supports the Penguard Penguin protocol.

\subsection{System components}

\begin{description}
    \item [Penguard Android Application] is an Android app required to use Penguard. The app can act as a guardian or as a penguin (guardian mode or penguin mode). When acting as a penguin, it does support the Penguard Penguin protocol. The user should be able to select what the app should do when the penguin is lost. Options include sounding an alarm and sending GPS coordinates to the guardians via SMS. When acting as a guardian, it provides an user interface that displays the status of each monitored penguin. The status includes the signal strength (RSSI), which of the other guardians see that specific penguin, and whether the penguin supports the Penguard Penguin Protocol.
    \item [Guardians] monitor the penguins using the Penguard app in Guardian mode. The term ``Guardian'' refers to the Penguard app in guardian mode.
    \item [Penguins] are bluetooth devices. They are registered with the guardians and then monitored. They optionally support the Penguard Penguin protocol, allowing them to take actions when they are lost.
    \item [Penguard Discovery Server] allows guardians to find each other more easily. Guardians can register with the Discovery Server. The server keeps a list of all Penguard groups. Guardians can request information on a specific group. The server will reply with all IP addresses of all guardians within that group, such that the new guardian can communicate with the other group members. Furthermore, guardians can register new groups on the server. When a guardian stops the Penguard service, it is deregistered at the server and removed from its group. The server will automatically purge empty groups. The Penguard Discovery Server should be an optional component. When no server is present, guardians exchange the necessary information via broadcast packets on the local network.
\end{description}

\subsection{Calibration}

The range of RSSI values varies between devices, so it should be possible for a guardian to calibrate itself. That is done in a guided process where the penguin is first brought close to the guardian and then slowly carried away.

\subsection{Penguard Guardian Protocol}

The Guardian Protocol allows guardians to form groups, and it allows guardians within the same group to communicate.

\subsubsection{Forming groups}

Groups can be formed either via a Penguard Discovery Server or using broadcast packets on the local network. Every group has an unique ID.

One guardian acts as the group's creator. When a discovery server is used, the creator will register the group with the server. The server will then notify it whenever a new member registers with the group. When no server is used, the creator will act as the server. It will broadcast the group information on the local network. Other guardians can pick up that information and register with the group.

The guardians that do not act as creators will have to register with an existing group. When a discovery server is present, they ask that server for information on that specific group by sending the group's ID. That ID must hence be known beforehand by the guardian. The server will find the according group and send the IPs of all registered guardians to the new guardian.

When no server is present, the guardian will listen for Penguard discovery packets on the local network. These packets contain the group ID. When the guardian decides to register with a certain group, it will ask the group's creator for the group information, and receive all group member's IP addresses from the creator.

\subsubsection{Communicating}

When a guardian is registered with a group, it will immediately start sending its status to the other group members. The status includes information about which of the penguins currently guarded it sees. It will also receive similar status updates from other group members.

When a guardian notices that its IP address changes, it will immediately notify all other guardians.

Guardians should also relay information to other guardians to provide some redundancy. When a link between two guardians fails but both can still reach a third guardian, the third guardian should act as a relay between the two.

\subsection{Penguard Penguin Protocol}

The Penguin protocol allows for penguins to detect when they are lost.

It is a Bluetooth Low Energy protocol.

The penguin should advertize that he supports the Penguin Protocol. It can be activated and deactivated.

Once activated, the penguin enters its active state. In this state, it will listen for Pings from the guardians and acknowledge them. The guardians must send these pings regularly. When the penguin does not receive a ping for long enough, it will consider itself lost and enter its lost state. Once it receives another Ping, it will transition back to active state.

When deactivated, the penguin will go in its inactive state. In that state, the penguin will reply to Pings saying that it is inactive.

The penguin can also tell the guardians which information it would like to receive from the guardians. The guardians will poll for this information once. They will then send the required information to the penguin.

The required information should not change.

\section{Requirements}

During this project, we will need the following hardware:

\begin{itemize}
    \item Several Android smartphones
    \item Several Bluetooth devices
    \item A server (kindly provided by VSOS)
\end{itemize}

We will need the following software:

\begin{itemize}
    \item Android Studio
\end{itemize}


\section{Work Packages}

\begin{itemize}
        \item {\bf WP1}: Implement the Penguard Guardian Protocol - communication part
        \item {\bf WP2}: Implement the Penguard Guardian Protocol - group finding part (no discovery server)
        \item {\bf WP3}: Write a Penguard service that can handle numerous Penguins without Penguin Protocol support
        \item {\bf WP4}: Write a Penguard Penguin service that implements the Penguard Penguin protocol (as a penguin)
        \item {\bf WP5}: Implement calibration functionality for guardians
        \item {\bf WP6}: Extend the Penguard service that can also handle Penguins with Penguin Protocol support
        \item {\bf WP7}: Write a Penguard Discovery Server supporting the Penguard Guardian protocol
        \item {\bf WP8}: Implement the Penguard Guardian Protocol - group finding part (using discovery server)
        \item {\bf WP9}: Design a functional graphical user interface for the Penguard app
\end{itemize}
 

\section{Milestones}

\subsection{Penguard Guardian Protocol - communication}

Guardians can communicate with each other, given that their IP addresses are provided beforehand. Status updates work.

\subsection{Penguard Guardian Protocol - group finding (local)}

Guardians can create and join groups via discovery on the local network.

\subsection{Discovery server}

Guardians can create and join groups via a discovery server. The server correctly handles the groups.

\subsection{Monitoring a single penguin not supporting the Penguin Protocol}

A guardian monitors one penguin and can detect when it is lost. 

\subsection{Monitoring multiple penguins not supporting the Penguin Protocol}

A guardian monitors multiple penguins and can detect when either of them is lost. 

\subsection{Monitoring multiple penguins supporting the Penguin Protocol}

A guardian correctly uses the Penguard Penguin protocol to detect whether penguins support the protocol, activate them, detect what information they require, send it to them, and ping them.

\subsection{Calibration}

Guardians have the option to calibrate. Calibration is stored locally.

\subsection{User interface}

The user interface is functional, understandable and polished.

% The following two commands are all you need in the
% initial runs of your .tex file to
% produce the bibliography for the citations in your paper.
\bibliographystyle{abbrv}
\bibliography{report}  % sigproc.bib is the name of the Bibliography in this case
% You must have a proper ".bib" file

%\balancecolumns % GM June 2007

\section{Deliveries}

We expect to deliver the following:

\begin{itemize}
    \item Code for the Penguard Android application
    \item Code for the Penguard Discovery Server
    \item Documentation for the Penguard Penguin Protocol
    \item Documentation for the Penguard Guardian Protocol
\end{itemize}

\end{document}
